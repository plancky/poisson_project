\documentclass[aspectratio=169]{beamer}
\usetheme{metropolis}
\metroset{numbering=fraction}
\usecolortheme{owl}
\usepackage{amsmath,fontenc,amsfonts,mathtools}%
\setbeamertemplate{section in toc}[sections numbered]
\setbeamertemplate{subsection in toc}[subsections numbered]

\makeatletter
\setbeamertemplate{section page}{
	\centering
	\begin{minipage}{22em}
		\raggedright
		\usebeamercolor[fg]{section title}
		\usebeamerfont{section title}
		\thesection.~\insertsectionhead\\[-1ex]
		\usebeamertemplate*{progress bar in section page}
		\par
		\ifx\insertsubsectionhead\@empty\else%
		\usebeamercolor[fg]{subsection title}%
		\usebeamerfont{subsection title}%
		\thesection.\thesubsection.~\insertsubsectionhead
		\fi
	\end{minipage}
	\par
	\vspace{\baselineskip}
}

\makeatother


\usepackage{graphicx}
\usepackage{svg}
\usepackage[backend=bibtex,style=chem-acs]{biblatex}
\addbibresource{careyfoster.bib}
\usepackage{csvsimple,longtable,booktabs}


\title{CAREY FOSTER's BRIDGE}
\subtitle{A Brief Review}
\author{Shashvat Jain}
\date{\today}
\begin{document}
	\maketitle
	\begin{frame}{OUTLINE}
		\tableofcontents
	\end{frame}
	\section{INTRODUCTION}
	\begin{frame}{What is Carey Foster's bridge? Why do we need another bridge?}
		The Carey Foster's bridge is another modification of Wheatstone bridge arrangement used to measure relatively medium or equal resistances compared to other resistances in the arrangement. As a result it becomes a necessity to ask why do we need Carey Foster's bridge?
		The answer to this question must lie within the subtleties of it's modification. \\
		So let's take a closer look with an experiment using the bridge.
	\end{frame}

	\section{THE EXPERIMENT}

	\subsection{AIM AND APPARATUS USED}

	\begin{frame}{AIM AND APPARATUS USED}
		\begin{itemize}
			\item \textbf{AIM : } To determine an unknown low resistance using the Carey Foster's bridge.
			\item \textbf{APPARATUS USED : } Carey foster's bridge, unknown low resistance,Resistance box, Battery, jockey, one way key, Galvanometer,small shunt resistance, connecting wires of almost zero resistance.
		\end{itemize}
	\end{frame}

	\subsection{CIRCUIT DIAGRAM}

	\begin{frame}{CIRCUIT DIAGRAM}
		\begin{figure}
			\centering
			\includegraphics[width=0.7\columnwidth]{Carey_Foster_bridge.png}
			\caption{The Carey Foster bridge. The thick-edged areas are busbars of almost zero resistance.\cite{dia1}}
			\label{fig:1}
		\end{figure}
	\end{frame}
	

	\subsection{THEORY}
	\begin{frame}{THEORY}
		X is the unknown resistance. P,Q and Y are known resistances of magnitude comparable to that of X, forming the other half of the bridge. The bridge wire EF has a jockey contact D placed along it and is slid until the galvanometer G measures zero. The thick-bordered areas are thick copper busbars of almost zero resistance.\\ 
		The bridge is said to be balanced when no current passes through the galvanometer.\\ 
		\begin{itemize}
			\item
		\end{itemize}
	\end{frame}

	\begin{frame}{FORMULA USED}
		Let $\ell_{1}$ be the length ED when X is placed on the right and
		 $\alpha$ is the unknown left-side extra resistance EX and $\beta$ is the unknown right-side extra resistance FY, and $\sigma$is the resistance per percent length of the bridge wire: 
		\begin{equation*}
			{\displaystyle {P \over Q}={{X+\sigma (\ell _{1}+\alpha )} \over {Y+\sigma (100-\ell _{1}+\beta )}}}
		\end{equation*}
		and add 1 to each side:
		\begin{equation}
			{\displaystyle {P \over Q}+1={{X+Y+\sigma (100+\alpha +\beta )} \over {Y+\sigma (100-\ell _{1}+\beta )}}}
		\end{equation}
		Now swap X and Y. $\ell_{2}$ is the new null point reading: 
		\begin{equation*}
			{\displaystyle {P \over Q}={{Y+\sigma (\ell _{2}+\alpha )} \over {X+\sigma (100-\ell _{2}+\beta )}}}
		\end{equation*}
	\end{frame}

	\begin{frame}{FORMULA USED}
		and add 1 to each side:
		\begin{equation}
			{\displaystyle {P \over Q}+1={{X+Y+\sigma (100+\alpha +\beta )} \over {X+\sigma (100-\ell _{2}+\beta )}}}
		\end{equation}
		From (1) and (2) we get:
		\begin{equation*}
			{Y+\sigma (100-\ell _{1}+\beta )=X+\sigma (100-\ell _{2}+\beta )}
		\end{equation*}
		\begin{equation}
			{\implies X= Y+ \sigma (\ell _{2}-\ell _{1})}
		\end{equation}
	\end{frame}
	\begin{frame}{Subtleties of Modification}
		\begin{itemize}
			\item Note that the unknown unwanted resistances $\alpha$ and $\beta$ have no affect on the finally obtained resistance $X$. This reduces the error in the result to a great extent.
			This boosts the sensitivity of the instrument 
			\item Comparing this to a metre bridge, This setup ensures that the components of the circuit are not majorly harmed incase the unknown resistance is very small.
			\item This enables more accurate measurements of smaller resistances.
		\end{itemize}
	\end{frame}
	\subsection{PROCEDURE}
	\begin{frame}{PROCEDURE}
		\textbf{To find the unknown resistance(X)}
		\begin{enumerate}
			\item Setup the circuit as shown in figure \ref{fig:1}
			\item Start by switching on the circuit and sliding the galvanometer jockey ubtil the deflections become very small.
			\item Now remove the shunt resistance and search for null point in the region of minimal deflection.
			\item Once the null point is found, measure EF.
			\item Swap X and Y and repeat the above steps to get $\ell_2$.
			\item Repeat 1,2,3 qnd 4 with different values of Y.
		\end{enumerate}
		\textbf{To find the resistance per unit length ($\sigma$)}
		\begin{enumerate}
			\item Setup the circuit as shown in circuit diagram but now replace X with a wire of zero resistance. 
		\end{enumerate}
	\end{frame}
	\subsection{OBSERVATIONS}
	\begin{frame}{DATA}
		Data was obtained from the following lab report \cite{read}
		\begin{enumerate}
			\item LC of the instrument used to measure Y = 0.01 $\Omega$
			\item LC of metre scale = 0.1cm
		\end{enumerate}
		\begin{table}[h]
			\caption{Observations for Resistance per unit length($\sigma$)}
			\csvreader[longtable=cccccc,
			table head=\toprule\bfseries S.No. &\bfseries Y($\Omega$) &\bfseries $\ell_{1}(cm) $ &\bfseries $\ell_{2}(cm) $ &\bfseries $\ell_{1} - \ell_{2}$ (cm) &\bfseries $\sigma$($\Omega / cm$) \\ \midrule\endhead\bottomrule\endfoot,
			late after line=\\,
			before reading={\catcode`\#=12},after reading={\catcode`\#=6}
			]{careyfoster_01.csv}{1=\Sno,Y=\Y,l2=\li,l1=\lii,l2-l1=\di,rho=\X}{\thecsvrow & \Y & \li & \lii & \di & \X}
		\end{table}
	
	\end{frame}
	\begin{frame}{DATA}
		\begin{table}[h]
			\caption{Observations for value of X}
			\csvreader[longtable=cccccc,
			table head=\toprule\bfseries S.No. &\bfseries Y($\Omega$) &\bfseries $\ell_{1}(cm) $ &\bfseries $\ell_{2}(cm) $ &\bfseries $\ell_{2} - \ell_{1}$ (cm) &\bfseries $X$($\Omega$) \\ \midrule\endhead\bottomrule\endfoot,
			late after line=\\,
			before reading={\catcode`\#=12},after reading={\catcode`\#=6}
			]{careyfoster_02.csv}{1=\Sno,2=\Y,3=\li,4=\lii,5=\di,6=\X}{\thecsvrow & \Y & \li & \lii & \di & \X}
		\end{table}
	\end{frame}
	\begin{frame}{DATA ANALYSIS}
		\begin{itemize}
			\item Mean value of $\sigma$ = 0.032 $\Omega/cm$
			\item Error in $\sigma$ due to LC $\approx 0.005$ 
			\item Standard error in $\sigma$ = 0.002
			\item Mean value of X = 0.7 $\Omega$
			\item Error in X due to LC $\approx 0.1$ 
			\item Standard error in X = 0.01
		\end{itemize}
	\end{frame}
	\subsection{RESULTS}
	\begin{frame}{RESULTS}
		\begin{itemize}
			\item Value of unknown resistance X = $0.7 \pm 0.1$
		\end{itemize}
	\end{frame}
	\begin{frame}{PRECAUTIONS}
		\begin{itemize}
			\item Make sure that all resistances are of low and comparable magnitude. High unknown resistances can render the galvanometer and resistance wire useless.
			\item Do not forget to to remove shunt resistance when close to the null point.
			\item Do not rub the jockey against the resistance wire wire, slide it gently. 
			\item Make a note of the error in the measurements of the known resistances.
		\end{itemize}
	\end{frame}
	\begin{frame}{Light Dependent Resistances LDRs}
		\begin{itemize}
			\item During the early 20th century the bridge was popularly used for measuring unknown resistances.
			\item Carey Foster's bridge arrangement is used in light detectors where the unknown resistance is a Light dependent resistance(LDR) whose resistance decreases with increase in the intensity of light falling on it.
		\end{itemize}
	\end{frame}
\section{REFERENCES}
	\begin{frame}[t]
	\frametitle{REFERENCES}
		\printbibliography[heading=none]
	\end{frame}
\end{document}
