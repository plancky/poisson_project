\documentclass[letterpaper,11pt]{article}
\usepackage[a4paper, total={6.5in, 11in}]{geometry}

\usepackage[utf8]{inputenc}
\usepackage[T1]{fontenc}

\usepackage[p,osf]{cochineal}
\usepackage[scale=.95,type1]{cabin}
\usepackage[cochineal,bigdelims,cmintegrals,vvarbb]{newtxmath}
%\usepackage{newpxmath}
\usepackage[zerostyle=c,scaled=.94]{newtxtt}
\usepackage{amsmath,accents}
\usepackage{relsize}
\usepackage{listings}
\usepackage{xcolor}
\usepackage{outlines}
\usepackage{graphicx}
\usepackage{caption}
\usepackage{subcaption}
\definecolor{greyish}{HTML}{798799}
\usepackage{hyperref}
\hypersetup{
	colorlinks=true,
	linkcolor=blue,
	urlcolor=darkblue,
}
%\usepackage[backend=biblatex,style=science]{biblatex} %Imports biblatex package
%\addbibresource{spect.bib} %Import the bibliography file

\usepackage[section]{placeins}
\pagestyle{empty}
%\titleformat{\section}{\vspace{-4pt}\scshape\raggedright\large}{}{0em}{}[\color{black}\titlerule \vspace{-5pt}]
\newcommand{\newsection}[1]{\section{\sffamily{\bfseries{#1}}}}
\newcommand{\myvec}[1]{\accentset{\rightharpoonup}{#1}}
\newcommand{\bd}[1]{\textbf{\textit{#1}}}

\begin{document}
	
	\begin{center}
		\sffamily{\bfseries{\huge SECC PROJECT PROPOSAL}} \\[4mm]
		\large{Computational techniques for solving the Poisson’s equation} \\[4mm]
		\large{\textcolor{greyish}{Team no.21}} \\[1mm]
		\large{\textcolor{greyish}{Shashvat Jain PHY1114, Brahmanand Mishra PHY1184, Akarsh Shukla PHY1216}}\\[3mm]
		\begin{abstract}
			\noindent
			Poisson's equation is a Second-order linear partial differential equation that is all around you, with its ability to model steady-state scalar fields such as gravitational and electric potential fields, temperature and pressure fields, as boundary value problems, it is often found in the toolbox of any physicist or engineer studying aerodynamics, thermal physics, electrostatics or magnetostatics. That’s not all, the same equation is used in geophysics, image processing, caustics engineering, stress and strain modeling, Markov decision processes, to name a few.\\ \noindent
			It would be a waste to let go of this opportunity to better understand Poisson's equation.  
		\end{abstract}	
	\end{center}
	
	\newsection{PROJECT SYNOPSIS}
	\noindent
	We have already solved the Poisson’s equation for electrostatic potential in our Electromagnetism paper last semester, although what we were taught was good enough to get us to the solution, it felt as if the equation held more than what we could see and there is only one way to find out, that is, by understanding the working of some numerical methods that bypass the difficult math of PDEs and analyzing the graphs that are obtained. Fortunately, we will do both in our project.\\
	The general form of Poisson's equation in Euclidean space is given here,
	\[
	\scalebox{1.3}{$\myvec{\nabla}^2 \varphi(\myvec{r}) = f(\myvec{r})$}
	\]
	where, $f(\myvec{r})$, $\varphi(\myvec{r})$ are the functions of position vector $\myvec{r}$. $f(\myvec{r})$ is given and $\varphi(\myvec{r})$ is sought. 
	\noindent

	\noindent
	We also aim to find how the analytical solution compares to the numerical solutions and explore the computational complexity of the numerical methods by studying the Electrostatic Poisson’s equation. 
	
	
	\newsection{PROBLEM}
	To analyse the potential between two capacitor plates kept small distance apart  at steady state. We will try to solve this problem using the finite differences method both in presence and absence of charges. We will use the following two methods to explicitly solve the  finite differeces method -:
	\begin{itemize}
		\setlength\itemsep{0.01mm}
		\item Gauss-Seidel Method 
	
		\item SOR Method
	\end{itemize} 

	
	
	\newsection{PLAN OF EXECUTION / ALGORITHM}
	Our plan of action has the following stages:-
	\begin{enumerate}
		\setlength\itemsep{0.01mm}
		\item {\bd{Translate the Physics to mathematics} - Convert the Physical model to a mathematical one.}
		\item {\bd{Do the mathematics} - We will study the analytic solution to the generated Poisson’s/Laplace equation.}
		\item \bd{Implement and execute } - We will study and implement the numerical methods in \textbf{python 3}.
		\item \bd{Understanding the graphs} - We hope to gain a better feel for the equation by visualising the solution obtained through computation.
		\item \bd{Explain what we have done} - Assimilation of knowledge is just the beginning, we hope to make you see what we gaine through the solution . 
	\end{enumerate}
	

	
	\newsection{OUTCOME}
	{We hope to demonstrate the use of finite difference method to solve the poisson and laplace equation and in process to doing so we would like to gain a deeper understanding of capacitor problem. \\[2mm] \noindent
	}
	{\bd{What we do not hope to achieve ?}\\[2mm] \noindent We would surely like to understand and analyse the result of our computation but we would not like to delve in the theoretical aspects of different methods involved in solving different partial equations due to the vastness of the topic and the limited nature of time.} \\[1mm]
	
	\newsection{TIMELINE}
	\begin{itemize}
		\item \bd{Week 1 to 3 - UNDERSTANDING PERIOD} - This will include stage 1 \& 2 from our plan of action.
		\item \bd{Week 4 to 5 - CODING PERIOD} - During this time we will code the numerical methods and graph the solutions for the report. (stage 3)
		\item \bd{Week 6 to 8 - FINAL REPORT GENERATION AND CLEANUP} - This period will be spent on the final report generation, this includes explaining all that we have learnt in the report. stage(4 \& 5)
	\end{itemize}	
	
		\newsection{SPECIFICATIONS}
	\begin{itemize}
		\setlength\itemsep{0.01mm}
		\item Programming language - Python 3
		\item Data visualization utility - Gnuplot 5.4 or above 
		\item Markup report generation using - \LaTeX  
	\end{itemize}
	\newsection{TENTATIVE CONTRIBUTIONS}
	\noindent {We have divided our work in three major parts-:}
	\begin{itemize}
		\item \bd{Coding } - most of the coding part will be done by Brahmanand and Shashvat 
		\item \bd{Plotting } - the plotting part will be done by Akarsh 
		\item \bd {report writing } - this part can be further divided into three parts -:
		\begin{itemize}
			\item\bd{Mehod for Computation} - this section we will be written by Brahmanand  
			\item\bd{Interpretation of Result} - This section will be written by Akarsh
			\item \bd{Error Analysis} - This section will be written by Shashvat
		\end{itemize}
	\end{itemize}
	
	Apart from these contributions, works like research for resources and other work will be done mutually.
	We would like to point out that above contributions are tentative and the finalisation of any part will done only after there is mutual consent from all three.
	
\end{document}
