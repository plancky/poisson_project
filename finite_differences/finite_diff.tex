\documentclass[letterpaper,11pt]{report}
\usepackage[a4paper, total={6.5in, 11in}]{geometry}
\usepackage{amsmath}
\setlength{\parindent}{-1pt}
\numberwithin{equation}{section}
\usepackage[utf8]{inputenc}
\usepackage[T1]{fontenc}
\usepackage{tabularx}
\newcommand{\pa}{\partial}

%%%%%%%%%%%
\begin{document}    
\section{Finite Differences}
Poisson's equation is given by,
\begin{equation}
    -\nabla^2 u(x,y) = \rho(x,y)    
\end{equation}

The laplacian in euclidean geometry is defined as the sum of second derivatives of the function $u(x,y)$ with each of the independent variables.
In 2D we have,
\begin{equation}
    \nabla^2 u(x,y) = \frac{\pa ^2 u}{\pa x^2} + \frac{\pa ^2 u}{\pa y^2} = u_{xx} + u_{yy}    
\end{equation}

{\Large Difference Quotient}

    Forward
    \[
        u_x = \lim_{h \rightarrow 0} \frac{u(x+h,y) - u(x,y)}{h}
    \] 
    Backward
    \[
        u_x = \lim_{h \rightarrow 0} \frac{u(x,y) - u(x-h,y)}{h}
    \]
    Centered
    \[
        u_x = \lim_{h \rightarrow 0} \frac{u(x+\frac{h}{2},y) - u(x-\frac{h}{2},y)}{h}
    \]

    Second Derivative
    \begin{align*}
        u_{xx} &= \lim_{h \rightarrow 0} \frac{u_x(x+\frac{h}{2},y) - u_x(x-\frac{h}{2},y)}{h} \\
        &= \lim_{h \rightarrow 0} \frac{\frac{u(x+h,y) - u(x,y)}{h} - \frac{(u(x,y)- u(x-h,y))}{h}}{h}\\
        &= \lim_{h \rightarrow 0}\frac{u(x+h,y)-2 u(x,y) + u(x-h,y)}{h^2}
    \end{align*}
    Considering h to be sufficiently small we obtain the finite Difference approximation introducing the error term that will be ascertained later,
    \begin{equation}
        u_{xx} \approx \frac{u(x+h,y)-2 u(x,y) + u(x-h,y)}{h^2}
    \end{equation}

    If Domain of the function is confined to a Region R and discritized by considering a set of points lying on the intersections of the square mesh given below,
    
    \begin{align*}
        u_{xx}(jh,kh)
    \end{align*}
    \\[2mm] the Finite Difference approximations are then applied to each and every point on the grid to obtain a system of linear equations which can then be solved explicity for less number of points or implicitly using iterative relaxation methods.
    \\[2mm]


    {\Large Error}
    
    
    
     
\end{document}
