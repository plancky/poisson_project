\documentclass{article}
\usepackage[utf8]{inputenc}
\usepackage{amsmath}

\title{ Successive Over Relaxation Method}

\begin{document}
\maketitle

\section{Introduction}

This method is a variant of \textbf{Guass- siedel law} for solving a linear system of equation by iterative method , resulting in faster convergence. The main purpose of this method is to solve the linear systems automatically on the digital systems.These methods were designed for computation by \textbf{human calculators}.\linebreak

\hspace*{5mm}\textit{This method was introduced by  \textbf{David M. Young Jr.} and by \textbf{Stanley P. Frankel} in $1950$}

\section{ Mathematic Form}

Suppose, we have $n$ numbers of linear equations with $x$ as unknown.
So , it can also be written as:
\begin{equation} \label{eq1}
AX = {b}
\end{equation}
where, \linebreak
\hspace*{10mm}
\[ A=
\begin{bmatrix}
a_{11} & a_{12} & a_{13} &  & . & . & . & a_{1n}\\
a_{21} & a_{22} & a_{23} & . & . & . & . & a_{2n}\\
a_{31} & a_{32} & a_{33} & . & . & . & . & a_{3n}\\
. & . & . & . & . & . & . & .\\ 
. & . & . & . & . & . & . & .\\
a_{n1} & a_{n2} & a_{n3} & . & . & . & . & a_{nn}\\ 
\end{bmatrix}
,
\] 
X=
$\begin{bmatrix}
x_{1}\\
x_{2}\\
x_{3}\\
.\\
.\\
.\\
x_{n}\\

\end{bmatrix}$
\hspace*{3mm}
and b=
$\begin{bmatrix}
b_{1}\\
b_{2}\\
b_{3}\\
.\\
.\\
.\\
b_{n}\\

\end{bmatrix}$\linebreak

Now breaking matrix A in $D$ (\textsc{ diagonal component}), $L$ (\textsc{Lower triangular matrix}) and  $U$ (\textsc{upper triangular matrix})\linebreak  
Therefore, A can be written as-
\begin{equation} \label{eq2}
A = D + L + U
\end{equation}
where,
\hspace*{10mm}
$D={\begin{bmatrix}a_{11}&0&\cdots &0\\0&a_{22}&\cdots &0\\\vdots &\vdots &\ddots &\vdots \\0&0&\cdots &a_{nn}\end{bmatrix}},\quad L={\begin{bmatrix}0&0&\cdots &0\\a_{21}&0&\cdots &0\\\vdots &\vdots &\ddots &\vdots \\a_{n1}&a_{n2}&\cdots &0\end{bmatrix}}$\linebreak

and
\hspace{8mm}
$\quad U={\begin{bmatrix}0&a_{12}&\cdots &a_{1n}\\0&0&\cdots &a_{2n}\\\vdots &\vdots &\ddots &\vdots \\0&0&\cdots &0\end{bmatrix}}.$

Now, equation (1) can  be written as -
\begin{equation} \label{eq3}
(D+\omega L)\mathbf {x} =\omega \mathbf {b} -[\omega U+(\omega -1)D]\mathbf {x} 
\end{equation}
Here , the constant $\omega$ is called the relaxation factor.\linebreak
As, we already mentioned that it is a iterative method so we can solve the left hand side of the above equation  for $\mathbf{x}$ using the previous value of $\mathbf{x}$ from the right hand side.

Now using \textbf{iteration } it can be written as :

$\mathbf {x} ^{(k+1)}=(D+\omega L)^{-1}{\big (}\omega \mathbf {b} -[\omega U+(\omega -1)D]\mathbf {x} ^{(k)}{\big )}=L_{w}\mathbf {x} ^{(k)}+\mathbf {c}$

where,\linebreak 
\hspace*{10mm}\textbf{$X^k$} is the $n^{th}$ approximation. \linebreak 
\hspace*{10mm}\textbf{$X^{k+1}$} is the next approximation to $n^{th}$. \linebreak

Now,by using \textbf{forward substitution} it can be written as:
\[
x_{i}^{k+1}=(1-\omega )x_{i}^{k}+{\frac {\omega }{a_{ii}}}\left(b_{i}-\sum _{j<i}a_{ij}x_{j}^{(k+1)}-\sum _{j>i}a_{ij}x_{j}^{(k)}\right)\]
\[where ,\quad i=1,2,\ldots ,n.
\]
\end{document}

