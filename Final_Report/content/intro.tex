	\section {Introduction}
    \subsection{Motvation}
    \noindent 
    We first encountered Laplace Equation during our course in electricity and magnetism in second semester and we were fascinated with how one can calculatge the potential in a region just by knowing the boundary condition, ofcourse the region has to be charge free for applying Laplace Equation. After Laplace Equation , we were introduced to Poisson Equation which we were able to solve for region having charges( or sources ). When we were given the oppurtunity to choose a project in our computational physics this semester, it did not take us long to decide the topic for project.
    \subsection{General Idea}
    \noindent
    In our project we will try to tackle the Laplace and Poisson equaton which is an ellipitic linear partial differential equation having application in various fields of physics ranging from thermodynamics, electrostatics etc. We will solve the equation computationally using the method of finite differences in one and two dimensions for rectangular membrane. We will first convert the partial differential equation into system of linear differential equation and then use three different iteration schemes for solving those system of linear equation.\\ 
    \subsection{Poisson and Laplace Equation}
    	The general form of Poisson's equation in Euclidean space is given here,
    \[
    \scalebox{1.3}{$\vec{\nabla}^2 \varphi(\vec{r}) = f(\vec{r})$}
    \]
    where, $f(\vec{r})$, $\varphi(\vec{r})$ are the functions of position vector $\vec{r}$. $f(\vec{r})$ is given and $\varphi(\vec{r})$ is sought. \\
    In our case  $\varphi(\vec{r})$ is electrostatic potential, $ f(\vec{r})$ is the charge density of region in which solution is required. In a special case in which  $f(\vec{r}) = 0 $ we can apply same process for solving laplace equation for a region free of charge.
    \subsection{Plan of Report}
    In theory section we will formulate and explain the physical problem we chosen to solve using finite difference method. In methodology we will explain our methods and explain the algorithm followed for programming. In the Result and anlaysis section we analyse our results compared to what we expect to obtain and also compare different iterative methods based on our computation relating problem.And in computation we would like to discuss our experiences and results in brief.