\section{Appendix}
\subsection{Non-dimensionalisation}
In this section we would like to non-dimensionalise Laplace equation. The general form of Laplace equation is:
		\begin{center}
			$$\mvec{\nabla}^2 \varphi(\mvec{r}) = 0$$
		\end{center}
		\noindent
		or 
		\begin{center}
			\[  
			\frac{\partial^2 u}{\partial x^2} + \frac{\partial^2 u}{\partial y^2} = 0
			\]
		\end{center}
		or 
		\begin{center}
			\[u_{xx} + u_{yy} = 0 \]
		\end{center}
	\noindent
	where \textbf{$ u_{xx} $} represents the second order partial derivative of u with respect to x.
	and $ u_{yy} $ represents second order partial derivative of u with respect to y.
	In our case $ u = V $ so we need to non -dimensionalise the following equation:
	\begin{center}
			\[V_{xx} + V_{yy} = 0 \]
	\end{center}
	\noindent 
	Now we will replace the dimensional variable with non-dimensional variables. Let \[\hat{V} = \frac{V}{V_s},  \hat{x} = \frac{x}{x_s},  \hat{y} = \frac{y}{y_s}\]
	\noindent
	here the $ \hat{V}, \hat{x}, \hat{y} $ are dimensionless variables and $ V_s, x_s, y_s$ are the scaling variables having the same unit as their counterparts.
	\[  
	\frac{\partial^2 V}{\partial x^2} = \frac{\partial}{\partial x} \Bigg({ \frac{\partial V}{\partial x}}\Bigg) \]
	\[ = \frac{\partial}{\partial x} \bigg(\frac{\partial (\hat{V}V_s)}{\partial (\hat{x}x_s)}\bigg) \]
	\[ = \frac{V_s}{x_s} \Big[{\frac{\partial}{\partial (x_s \hat{x})}} \Big({\frac{\partial \hat{V}}{\partial \hat{x}}}\Big)\Big]\]
	\[ = \frac{V_s}{x_s^2} \Big[{\frac{\partial^2 \hat{V}}{\partial \hat{x^2}}}\Big]\]
	\noindent
	similarly, we can write for y as : 
	\[\frac{V_s}{y_s^2} \Big[{\frac{\partial^2 \hat{V}}{\partial \hat{y^2}}}\Big]\]
	\noindent
	Now we can write our equation as:
	\[ = \frac{V_s}{y_s^2} \Big[{\frac{\partial^2 \hat{V}}{\partial \hat{y^2}}}\Big] + \frac{V_s}{x_s^2} \Big[{\frac{\partial^2 \hat{V}}{\partial \hat{x^2}}}\Big] = 0 \]
	So now if choose same magnitude for our scaling factor then we write our equation as:
	\[ = \frac{V_s}{x_s^2} \Big[{\frac{\partial^2 \hat{V}}{\partial \hat{y^2}}} +  {\frac{\partial^2 \hat{V}}{\partial \hat{x^2}}}\Big] = 0 \]
	\[ = {\frac{\partial^2 \hat{V}}{\partial \hat{y^2}}} +  {\frac{\partial^2 \hat{V}}{\partial \hat{x^2}}} = 0 \]
	Hence so if we just use the same sacling factors for both x and y variable then we don't need to non-dimensionalise the Laplace equation.
	\subsubsection{Poisson Equation}
	\noindent
	Now since the general form of Poisson Equation is as follows:
	\begin{center}
		$$\mvec{\nabla}^2 \varphi(\mvec{r}) =  f(\mvec{r})$$
	\end{center}
	\noindent
	or 
	\begin{center}
		\[  
		\frac{\partial^2 u}{\partial x^2} + \frac{\partial^2 u}{\partial y^2} = - f({r})
		\]
	\end{center}
	
	So for our case we can write it as:
	\[  
	\frac{\partial^2 V}{\partial x^2} + \frac{\partial^2 V}{\partial y^2} =  -\frac{\rho}{\epsilon_0}
	\]
	Now following the same procedure  for LHS as done in case of Laplace equation we get:
		\[  \frac{V_s}{x_s^2} \Big[{\frac{\partial^2 \hat{V}}{\partial \hat{y^2}}} +  {\frac{\partial^2 \hat{V}}{\partial \hat{x^2}}}\Big] =   \frac{-\rho}{\epsilon_0} \]
	or
		\[  {\frac{\partial^2 \hat{V}}{\partial \hat{y^2}}} +  {\frac{\partial^2 \hat{V}}{\partial \hat{x^2}}} =   \frac{-\rho}{\epsilon_0} \frac{x_s^2}{V_s} \]
		Now we will choose the scaling factors such that the value of RHS becomes unity to ease our computation part and it will depend upon the question.
		
\subsection{Code}
\subsubsection{Python Scripts}
The file \textit{poisson.py} containing all our important methods and classes.
\begin{lstlisting}[language=Python]
from numba import jit
import numpy as np
import time as t
import os

def convert_gnuplotgrid3d(FILE):
    f = open(FILE,'r')
    prev_x = f.readline().split()[0]
    lst1= ""
    for line in f :
        new_x = line.split(' ')[0]
        if new_x!=prev_x:
            lst1+="\n"+line
            prev_x = new_x
        else:
            lst1+=line
    f.close()
    f2 =open(FILE,"w")
    f2.write(lst1)
    f2.close()

class mesh:
    def __init__(self,x=(0,1),y=(0,1),h=0.1,gtype="2D"):
        self.x_dim,self.y_dim=int((x[1]-x[0])/h +1),int((y[1]-y[0])/h +1) 
        self.y_dom=np.linspace(y[0],y[1],self.y_dim)
        self.x_dom=np.linspace(x[0],x[1],self.x_dim)
        self.h = h
        self.omega = dict()
        self.gtype= gtype 
        self.X,self.Y = np.meshgrid(self.x_dom,self.y_dom)
        if self.gtype == "1D":
            self.grid=np.ones((self.x_dim,),dtype="double")
        elif self.gtype == "2D":
            self.grid=np.ones((self.y_dim,self.x_dim),dtype="double")

    def dirichlet(self,x,excluded = "y"):
        if self.gtype=="2D" and len(x)==4:
            if excluded != "y":
                self.grid[0,:],self.grid[-1,:] = x[2] , x[3]
            if excluded !="x":
                self.grid[:,0],self.grid[:,-1] =  x[0] , x[1]
        elif self.gtype == "1D" and len(x)==2:
            self.grid[0],self.grid[-1] = x[0] , x[1]
        else:
            raise ValueError("Expected {0} but recieved {1} Boundary conditions.".format(self.gtype[0],len(x)))
    def get(self):
        return(self.grid)
    def set_loc(self,loc):
        self.loc = os.getcwd()+f"/dats/{loc}"
        try:
            os.mkdir(self.loc)
        except:
            pass
    def jacobi_poisson2d(self,p,nu=1,rtol=None):
        solve = jacobi2d(self.grid,self.h,p,rtol)
        self.u = solve[0]*nu
        self.dat = np.array([self.X.flatten('F'),self.Y.flatten('F'),self.u.flatten('F')],dtype =float)
        np.savetxt(f"{self.loc}/jacobi_poisson2d_{self.h}.dat",self.dat.T,fmt="%.20g")
        convert_gnuplotgrid3d(f"{self.loc}/jacobi_poisson2d_{self.h}.dat")
        return(solve)
    def sor_poisson2d(self,p,w,nu=1,rtol=None):
        solve = sor2dpoisson(self.grid,self.h,w,p,rtol)
        self.u = solve[0]*nu
        self.omega[w] = solve[1]
        self.dat = np.array([self.X.flatten('F'),self.Y.flatten('F'),self.u.flatten('F')],dtype =float)
        np.savetxt(f"{self.loc}/sor_poisson2d_{self.h}_{w}_{nu}.dat",self.dat.T,fmt="%.20g")
        convert_gnuplotgrid3d(f"{self.loc}/sor_poisson2d_{self.h}_{w}_{nu}.dat")
        return(solve)
    
    def save_omega(self):
        np.savetxt(f"/home/planck/Desktop/secc_project_proposal/dats/{self.loc}/omega_variation.dat",
        np.array([list(self.omega.keys()),list(self.omega.values())]).T)

@jit("Tuple((f8[:,:],f8))(f8[:,:],f8,f8,f8[:,:],f8)",nopython=True,cache=True)
def sor2dpoisson(x,h,overcf=1.9,p=None,rtol=None):
    k,m = x.shape[0],x.shape[1]
    if p is None :
        p = np.zeros(x.shape)
    if rtol is None :
        rtol = h**2
    for f in np.arange(1,1e+6,1):
        new_x= x.copy()
        for i in np.arange(0,k,1):
            for j in np.arange(1,m-1,1):
                left = new_x[i][j-1]
                right = new_x[i][j+1]
                ##Neumann 0 wrt y-axis at y_upper and y_lower bounds 
                if i == k-1 :
                    up = new_x[i-1][j]
                else :
                    up = new_x[i+1][j]
                if i == 0 :
                    down = new_x[i+1][j]
                else:
                    down = new_x[i-1][j]
                new_x[i][j] = new_x[i][j] + ((up+down+right+left + h**2*p[i][j])/4 - new_x[i][j]) * overcf
        er = np.abs((new_x - x )/ new_x).max()
        if er<=rtol: 
            break
        else:
            x = new_x.copy()
    return(new_x,f)

@jit("Tuple((f8[:,:],f8))(f8[:,:],f8,f8[:,:],f8)",nopython=True)
def jacobi2d(x,h,p=None,rtol=None):
    k,m = x.shape[0],x.shape[1]
    if p is None :
        p = np.zeros(x.shape)
    if rtol is None :
        rtol = h**2
    for f in np.arange(1,1e+6,1):
        new_x= x.copy()
        for i in np.arange(0,k,1):
            for j in np.arange(1,m-1,1):
                left = x[i][j-1]
                right = x[i][j+1]
                ##Neumann 0 wrt y-axis at y_upper and y_lower bounds 
                if i == k-1 :
                    up = x[i-1][j]
                else :
                    up = x[i+1][j]
                if i == 0 :
                    down = x[i+1][j]
                else:
                    down = x[i-1][j]
                new_x[i][j] = ((up+down+right+left + h**2*p[i][j])/4) 
        er = np.abs((new_x - x) / new_x).max()
        if er<=rtol: 
            break
        else:
            x = new_x.copy()
    return(new_x,f)

if __name__ == "__main__":
    pass

\end{lstlisting}
The file \textit{cap.py} utilizing all methods and classes to solve the main problem.
\begin{lstlisting}[language=Python]
from lib.poisson import *
import numpy as np
import time as t

'''Set constants'''
ep = 8.854e-12
us = 1
xs = 1
'''Set Domain'''
mm = mesh((0,4),(0,4.4),0.1)
h = np.diff(mm.x_dom)[0]
'''Set Dirichlet Boundary value conditions''' 
'''
------------upper_y_bound-----------
lower                           upper
_x_                             _x_
bound                           bound
------------lower_y_bound-----------
'''
#lower_bound_y = np.ones((mm.x_dim,))
#upper_bound_y = np.ones((mm.x_dim,))
lower_bound_x = np.ones((mm.y_dim,))*5/us
upper_bound_x = np.ones((mm.y_dim,))*5/us
mm.dirichlet([lower_bound_x,upper_bound_x,None,None])

'''Set charge distributiion'''
distri = np.zeros(mm.grid.shape)
xr,yr = np.round_(mm.X,2),np.round_(mm.Y,2)
bool_A = ((xr==2) | (xr==1) | (xr==3)) & (yr <=4) 
bool_B = np.any([xr==0.5,xr==1.5,xr==2.5,xr==3.5],axis=0) & (yr >= 0.4) 
distri[bool_A] = 2*1e+5*xs**2/us
distri[bool_B] = -2*1e+5*xs**2/us

'''Create a folder where data gets stored'''
mm.set_loc("t2")

#Solve poisson
t1= t.time()
s1=mm.sor_poisson2d(distri,1.9,nu=us,rtol=1e-8)
print(s1[1])
t2= t.time()
s2=mm.sor_poisson2d(distri,1,nu=us,rtol=1e-8)
t3= t.time()
print(s2[1])
s3=mm.jacobi_poisson2d(distri,us,rtol=1e-8)
t4= t.time()
print(s3[1])

#mm.save_omega()
print(t2-t1)
print(t3-t2)
print(t4-t3)

# Variation of iterations with omega
iter = []
for i in np.arange(1,2,0.01):
    iter.append(mm.sor_poisson2d(distri,i,nu=us,rtol=1e-8)[1])
np.savetxt("omega_variation.dat",np.array([np.arange(1,2,0.01),iter]).T)
    
\end{lstlisting}
\subsubsection{Gnuplot Scripts}
For plotting variation of no of iterations with $omega$, 
\begin{lstlisting}[language=Gnuplot]
set title "Variation of iterations with omega h=0.1, rtol=1e-8"
set xlabel "omeg "
set ylabel "No. of iterations for convergence, rtol = e-8"
set terminal pngcairo
set terminal png font arial 30 size 2400,1300
set xrange [1:2]
set ytics 200
set xtics 0.1
set grid xtics,ytics 
set output 'omega.png'
plot 'dats/omega_variation.dat' notitle w lp    
\end{lstlisting} 
The following script plots the colourmap of the potential scalar field in region $\Omega$,
\begin{lstlisting}[language=Gnuplot]
set title "Successive Over-relaxation Method h = 0.0125"
set xlabel "x "
set ylabel "y "
set zlabel "U (Volts)"
set terminal pngcairo
set terminal png font arial 32 size 2400,1300
set xrange [0:4]
set yrange [0:4.4]
set samples 80,88
set isosamples 80,88
set view 50,50
set pm3d at bs; set palette rgbformulae 30,31,32
unset surface
set view map
set output 'sor_map.png'
splot 'dats/dataset4_e-8/sor_poisson2d_0.0125_1_1.dat' notitle 
\end{lstlisting}
