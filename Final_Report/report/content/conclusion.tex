\section{Conclusion}
\subsection{Result}
We tried to solve the problem of interleaved capacitor using the finite difference method for solving poisson equation. We have solved the problem using three different iterative schemes. After completing this project we can say that the method of SOR is best for solving the system of linear equation after using the finite differences method. Using SOR method we were able to solve mesh of meshsize $ 0.0125 $ in just $ 22163 $ number of iteration and just $  54.35 \pm 0.1 sec $ seconds. Also we gained better insight and intuition after solving the problem of interleaved capacitor.
\subsection{Experience}
We have learnt a lot of new things during this project. We have learnt  how to solve a physical problem computationally and the various processes  it  involves such as non-dimensionalisation, the concept of convergence etc. This project has also boosted our python, latex and gnu skills. It has also increased our fascination with power of computation methods which allow us to solve complex physical problems without going through tedious calculatoins just by following some standard methods. 
 We also spend a good portion of time studying about theoretical aspects of different computational methods and trying to understand the concepts related to convergence, truncation error. This project has been an incredible journey for us beacause it has not only increased our theoretical , physical and compuational knowledge but it has also taught us about the importance of perseverance and patience as there were many topic  that we didn't understand easily just by studying about it from one or two places and  things that were not easily available in comprehendable manner for us due to advance nature of partial differential equation, sometimes we had to spend a lot of time in just trying to find a good resource. We would like to end this report by saying that we are very grateful that we had chosen such a topic that has taught us so much. 
